\section{Introduction}

The diffusion equation has been used to described many scientific phenomena for decades and is important to systems biology, in particular with respect to pattern development and the emergence of periodic structures from non-periodic sources during embryogenesis ~\cite{Kondo2010}. Because of its long history and extensive applications, mathematical studies have revealed the conditions for various patterns to arise. The inclusion of biomechanics to chemical kinetic frameworks naturally leads to non-diffusion like PDEs. Here we will show that our model used to study cellular blebbing behavior (see Chapter ~\ref{chap:cellbleb}) falls into this category. We believe a broad class of cellular behaviors obey similar mechanical constraints in conjunction with chemical dynamics, and therefore a study of the properties of these types of equations is a promising endeavor, and we plan to do a thorough literature review as part of our future work to identify such systems. In particular, we have chosen to investigate the conditions allowing for travelling wave solutions of a particular class of equations arising in cellular biophysics, a property already established in reaction-diffusion systems and sometimes called the Maxwell condition ~\cite{Anonymous:OS1MPwCl,Mori:2008hj}. We seek an analog of the Maxwell condition for the bleb model, and similar non-local PDEs. We will use our previously described cellular blebbing model as a test case.\\







