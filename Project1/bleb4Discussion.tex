% !TEX root = Project1.tex


%%%%%%%%%%%%%%%%%%%%%%%%%%%%%%%%%%%%%%%%%%%%%%%%%%%
\section*{Discussion}
%%%%%%%%%%%%%%%%%%%%%%%%%%%%%%%%%%%%%%%%%%%%%%%%%%%



%Excitability as recurrent theme in Cell Bio
Excitability is a recurrent theme in cell biology \cite{Huang:2013ge,Xiong:2010fb,FitzHugh:1961il, bement2015activator}. We find that the conditions for excitability emerge naturally from the mechanical properties of the cell surface, namely: the combination of a contractile cortex, a membrane exposed to internal hydrostatic pressure, and force-sensitive adhesions connecting them. In addition, membrane mechanical properties (i.e. surface tension) are sufficient for this excitability to lead to either limited-growth stationary blebs that heal in place, or traveling blebs reminiscent of circus movement. Notably, three classes of dynamics arise from the same model at different parameters: Stable, non-blebbing states (Fig.~\ref{fig::stationary}B), stationary blebbing (Fig.~\ref{fig::stationary}A,C); and traveling blebs (Fig.~\ref{fig::travel}A,B). Thus our model provides quantitative conditions for bleb growth and whether the bleb heals locally or travels. 

The model makes two main contributions. First, it allows elucidation of the determinants of the travel velocity in terms of biophysical parameters such as membrane tension and adhesion kinetics, Eq.~\ref{eq::dimVelocity}. Surprisingly, we find that hydrostatic pressure and myosin contractility only weakly determine velocity, while strongly determining other features such as whether the bleb forms, and its height. This is in distinction to previous assumptions \cite{Lim:2012fz} and other traveling waves in biology \cite{Allard:2012if}. 

%Why don't blebs spread as target patterns? Graded Radial Extension, membrane tension must spatially vary
Our second finding is that known biophysical mechanisms are insufficient to account for the compactness of traveling blebs in 3D. The excitability inherent in the system leads to traveling waves. However, a striking distinction from other excitable waves on a two dimensional domain is that other waves create bull's eye patterns or spiral patterns. Since local membrane-cortex detachment promotes nearby detachment symmetrically, why do blebs travel in a compact shape, rather than spreading in all directions? Generically, for a shape to remain approximately constant as it travels, the normal velocity on its perimeter must vary from maximal at its front to zero at its sides. This observation, termed the Graded Radial Extension condition \cite{Lee:1993bt}, was stated for steady cell motility but holds in general and therefore must be true for compact traveling blebs. One hypothesis we find sufficient to maintain compact travel is heterogeneity in the biophysical properties of the cell surface, such as adhesion density. There is no direct evidence that such heterogeneity is responsible for determining bleb travel paths, and it is likely that other mechanisms can explain compact travel. Since membrane tension is a strong determinant of local expansion velocity, it is possible that a model including different non-uniform membrane tension can recover a compact bleb in the absence of parametric heterogeneity. Other alternatives are: constraints set by lipid flow through the neck of the bleb \cite{Rangamani:2013ce}, or nematic ordering in the cortex \cite{Kapustina:2013gc}, which would break isotropic symmetry. For cells adhered to a rigid surface, the curvature is higher at the cell perimeter. This higher curvature could also potentially bias bleb formation and travel. We anticipate these will be a future direction of research. 

%Tension may increase in the bleb, but tension is not what limits bleb's outward expansion.
%We explored several assumptions about membrane tension dynamics, including assuming that local cortical contractility increases tension. While these have  

%If the bleb is bigger than the ablation, pressure must be partly local. -- NOT TRUE?

% Cytoplasmic actin and normal stress

A crucial feature of our model is the presence of a normal stress generated by the cortex, in addition to tangential stresses. We find that this normal stress is necessary for the dynamic healing and retraction of a traveling bleb. If myosin in the cortex generates an isotropic contractile stress, then it will induce stress in any direction in which there is F-actin. There is significant F-actin beneath the cortex (around 60\% of the density in the cortex \cite{Moeendarbary:2013bs,Clark:2013ef}), which is referred to as the cytoplasmic actin network and plays a role in cell integrity \cite{Luo:2013}. Our results suggest it also plays a role in retracting cellular protrusions.


% Poroelasticity

The rheology of the cytoplasm, which determines how pressure propagates, is under intense investigation. Our model assumes  a particular relationship between pressure change and volume change. To be as faithful to the correct rheology as possible, in the Results we simulate two extremes. Either (1) pressure relaxes entirely locally, with pressure at nearby locations unchanged, except perhaps on longer timescales if the bleb doesn't retract, i.e., pressure is local on short timescales, as described by our model Eq.~\ref{eq::localPressure}. Or, (2) pressure relaxation spreads rapidly, and it nearly equal everywhere following blebbing, i.e., pressure is global on short timescales, as described by our model Eq.~\ref{eq::globalPressure}. Recent computational models of detailed cell rheology \cite{Strychalski:HizQv1Ti} demonstrate a more complicated possibility. Assuming the cytoplasm is poroelastic \cite{Charras:2009dp, Moeendarbary:2013bs}, they find that, following blebbing, there is a small global drop in pressure, but full global equilibration is significantly slower. In the language of our model, this means that, on the $\sim1\s-10\s$ timescale we consider, part of the pressure drop is local and part is global. We may therefore be interested in a part-local, part-global pressure model. In Appendix ~\ref{sec:project1}, we consider pressure models in which local membrane protrusion leads to both local and global pressure drops. We find that pressure must be at least partly local, i.e., that neighboring regions are not equilibrated as quickly as at the site of protrusion, for blebbing to arise. As the global pressure drop is increased (corresponding to the assumption that the cytoplasm is less poroelastic and more like an incompressible fluid), the simulation approaches the purely global pressure model shown in Fig.~\ref{fig::variants}A.   


%Mathematical biology
Increasingly, mechanics is included in theoretical models of cellular processes \cite{Paszek:2015it,Thon:2012dh,Peleg:2011fz,Dobrowsky:2010dr,Qi:2006ez}. In these cases and others, subcellular mechanics equilibrates on sub-second timescales but drives processes that play out over seconds or slower, therefore mechanics is included via instantaneous force-balance or, equivalently, minimization of an energy functional as in Eq.~\ref{eq::energy} at every moment in time. Instead of reaction-transport (diffusion or advection) partial differential equations, these models can be expressed as a boundary value problem at each moment in time coupled to local time-dependent governing equations. This distinct class of models presents new opportunity for mathematical development. 
% Excitable waves in non-reaction-diffusion, non-local Maxwell condition 
For excitable reaction-diffusion systems, a straightforward condition termed the Maxwell Condition \cite{Mori:2008hj,Anonymous:OS1MPwCl,Murray:ur} can be computed that determines whether the excitation will generate traveling waves. Analogous conditions for the new class of mechanical models may exist, and will be the subject of future research. 

% Specific experiments to test predictions
Our model makes several testable predictions about how bleb behavior will be modulated by experimental perturbations. The specific predictions about bleb formation and travel velocity, in Results, correspond to changes in hydrostatic pressure, which can be modulated via the extracellular pressure by, e.g., osmolites; Cortical turnover, which can be promoted or slowed by jasplakinolide or cytochalasin-D \cite{Clark:2013ef, Sedzinski:2011ef}; Myosin contractility, which in blebs has been demonstrated to be susceptible to blebbistatin and indirectly to Y-compound \cite{Tinevez:2009bh}. In addition to these experiments, our model predicts that the ``reach" of the adhesion molecules, $\delta$, influences bleb characteristics via the (non-dimensional parameter $D$). It might be possible to modulate this parameter by mutagenically elongating or truncating cortex-membrane adhesion molecules. 

% Extensions
In addition to the model variants we explored here, this model is readily extendible to different surface geometries and assumptions about stresses below and above the cell surface. An intriguing direction of research is the coupling of the present model of surface mechanochemistry with different rheological models of how stress evolves inside the cell \cite{Strychalski:HizQv1Ti, Charras:2009dp}. Another direction is the coupling to extracellular fluid dynamics, which have recently been proposed to play a role in determining membrane dynamics, even on slow ($\sim 1\s$) timescales \cite{Anonymous:roDfYIJA}.


