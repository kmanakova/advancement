\chapter{Summary}

Cellular blebs

Blebs are pressure driven cell protrusions implicated in cellular functions such as cell division, apoptosis, and cell motility including motility of protease inhibited cancer cells. Because of their mechanical nature, blebs inform us about general cell surface mechanics including membrane dynamics, pressure propagation throughout the cytoplasm, and the architecture and dynamics of the actin cortex. Mathematical models including detailed fluid dynamics have previously been used to understand bleb expansion. Here we develop mathematical models on longer timescales that recapitulate the full bleb life cycle, including both expansion and healing by cortex reformation in 2D and 3D, in terms of experimentally accessible biophysical parameters such as myosin contractility, osmotic pressure, and turnover of actin and ezrin. The model provides conditions under which blebbing occurs, and naturally gives rise to traveling blebs. The model predicts conditions under which blebs travel or remain stationary, and predict the bleb traveling velocity, a quantity that has remained elusive in previous models.  As previous studies have used blebs as reporters of membrane tension and pressure dynamics within the cell, we have used our system to investigate various pressure equilibration models and dynamic, non-uniform membrane tension to account for the shape of a traveling bleb. We also find that traveling blebs tend to expand in all directions unless otherwise constrained, suggesting the importance of cell surface heterogeneity. 