\chapter{Abstract}

It is becomingly increasingly obvious that for particular cellular functions, biomechanics plays a central role. We use a combination of biomechanics and chemical reaction modelling frameworks to study biological processes at the cellular level. In particular, we are are studying the mechanical properties of the filamentous networks on the inner face of biological membranes. In the case of cell membranes, an underlying actin cortex acts to maintain cell shape and rigidity. Disruptions in the cortex lead to transient membrane protrusions known as blebs, which are implicated in a variety of cellular functions. Here, we developed a model which recapitulates the bleb life cycle and provides conditions under which blebbing occurs. Futhermore, our model can give rise to travelling blebs, a mysterious behaviour observed in some cell lines, and predicts travelling velocity, which had not been established by other models. The fast-slow dynamics of the model leads to interesting mathematical behaviours, such as a canard explosion associated with a supercritical hopf-bifurcation, which we plan to characterize in greater detail in the future. We have also derived a previously unknown necessary condition for travelling wave solutions to exist in such a system, and demonstrate sufficiency numerically. Nuclear blebs, as distinct from cellular blebs, are nuclear shape deformations often resulting from mutations in the gene encoding for lamin A/C, a major component of the filamentous network underlying the nuclear envelope. The specific defect in the mutatutes lamin proteins is unknown, as is the mechanism by which nuclear blebs form. We are developing mathematical models of the nuclear lamina in order to probe various portential defects in lamin proteins and correlate the resulting simulated nuclear shapes with patient and control data obtained from our collaborators in the Department of Biomedical Engineering  and the School of Medicine. 

