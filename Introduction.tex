\chapter{Introduction}

It is known that both chemical dynamics and mechanics play important roles in biological systems ~\cite{Wang2010}. One approach to studying the interplay between the two is through mathematical modelling. The application of mathematical models to the field of biology has a long history but with more recent technological advances, e.g. in computational power, the type of models we can build and simulate are becoming ever more complex and holistic.  Similarly, with renewed focus on bridging the gaps between physics, mathematics and biology, more elaborate biophysical model have been developed and simulated. Here, we developed biophysical models to study the membrane mechanics of eukaryotic cells and the nuclear lamina in animal cells. 

The eukaryotic cell membrane acts as the main barrier between the cell and the surrounding environment. Inside the cell, the membrane is supported by a dense networks of actin filaments call the cortex which is attached to the cell membrane via adhesion molecules. The cortex maintains the cell shape and contracts inward generating a internal pressure in the cell. Local disruptions to the cortex result in dynamic, transient protrusions of the cell membrane termed blebs. Blebs are implicated in many cellular activities including apoptosis, mitosis and motility, yet little is known about the mechanism underlying bleb formation. One particular aspect which has received little attention is the travelling behaviour of some blebs around the cell periphery, which may have implications in bleb based motility. We use mathematical modelling to elucidate the mechanisms underlying this behaviour. 

The equations arising from this study are interesting in their own right, as little research has been done for this class of integro-PDEs. More typical systems of equations arising in biological contexts, such as reaction-diffusion equations, have been studied extensively and many of their properties are well established. One such property is the existence of travelling wave solutions, which is known to be important to many biological processes. In the case of reaction-diffusion systems, the condition which allows for travelling wave solutions is known, whereas in our case of a non-diffusion like system, it is not. We believe the class of equations which our system falls into can and will be used to describe other biological systems in the future and so we seek an analog to this which will be applicable to a particular class of integro-PDEs. 

A superficially similar problem to that of cellular blebs is seen the nuclei of animal cells. Nuclei are bounded by a double membrane, and this double membrane is supported by a dense networks of lamin filaments termed the nuclear lamina. In cases where the gene responsible for producing lamin A/C, the main constituent of the nuclear lamina, is mutated, a  variety of disease states may ensue, including extremely severe diseases such as progeria and heart disease. The nuclei in these mutated cells often have a irregular shapes, with one or more protrusions termed nuclear blebs. In this case, the blebs are not due to a disruption in the linkage between the nuclear membrane and the lamina  (the lamina remains present in nuclear bleb region), but rather might be due to a defect in the lamina itself. The mechanism by which nuclear blebs form is unclear. We use mathematical modelling to probe potential defects in the lamin A/C protein, and work in conjunction with the Grosberg and Zaragoza labs to try to match simulated nuclei to their patient and control samples. This will elucidate the mechanism by which nuclear blebs form in disease states, and further our understanding of the disease itself. 






  